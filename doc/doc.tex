
\documentclass[a4paper,UKenglish,cleveref, autoref, thm-restate]{lipics-v2019}
%This is a template for producing LIPIcs articles. 
%See lipics-manual.pdf for further information.
%for A4 paper format use option "a4paper", for US-letter use option "letterpaper"
%for british hyphenation rules use option "UKenglish", for american hyphenation rules use option "USenglish"
%for section-numbered lemmas etc., use "numberwithinsect"
%for enabling cleveref support, use "cleveref"
%for enabling autoref support, use "autoref"
%for anonymousing the authors (e.g. for double-blind review), add "anonymous"
%for enabling thm-restate support, use "thm-restate"

%\graphicspath{{./graphics/}}%helpful if your graphic files are in another directory

\bibliographystyle{plainurl}% the mandatory bibstyle

\title{Documentation of float-smt}

\author{Felix Griesau}{Karlsruhe Institute of Technology, Germany}{TODO}{}{}
\author{Marc Huisinga}{Karlsruhe Institute of Technology, Germany}{mhuisi@protonmail.com}{}{}

\authorrunning{F. Griesau and M. Huisinga}

\Copyright{Felix Griesau and Marc Huisinga}

\ccsdesc[500]{Theory of computation~Automated reasoning}

\keywords{Automated theorem proving, SMT solving, Z3, Floating point, IEEE 754, smtlib}

\category{}

\relatedversion{}

\supplement{} %TODO: add github repo

\acknowledgements{}

\nolinenumbers

\hideLIPIcs 

\begin{document}

\maketitle

\begin{abstract}
This report documents the float-smt library, which implements a floating point theory using Z3. 
\end{abstract}

% Use \autoref for references and lstlisting for code.

\section{Introduction}
% TODO: motivate the project

\section{Usage instructions}
% TODO: installation, usage and testing instructions

\section{Technical environment}
% TODO: describe stuff like python, z3, z3py and our development environment

\section{Project structure}
% TODO: superficially explain the different components of the library (sorts, operations, packing etc.) and the file/module organization. shortly explain api.py and its relation to the other stuff so we don't need to explain api.py later.

\section{Implementation details}
% TODO: briefly note which functions were implemented and go into detail for interesting functions. explain how interesting functions were implemented, but also why certain design decisions were made.

\subsection{Sorts and constructors}
% TODO: sorts.py, constructors.py, converter.py

\subsection{Utilities and predicates}
% TODO: utils.py, conversions.py, predicates.py

\subsection{Pre- and postprocessing}
% TODO: packing.py

\subsection{Operations}
% TODO: operations.py

\section{Testing methodology}
% TODO: explain what was tested in test.py and how

\section{Experimental evaluation}
% TODO: evaluate our solver against z3 (and others?) and explain why it performs terribly and how one might improve it

\section{Conclusion}
% TODO: summarize the project and evaluate the outcome, putting emphasis on the correctness verified against z3

\bibliography{doc}

\appendix

\end{document}
